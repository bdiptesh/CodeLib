\documentclass[titlepage]{article}
\usepackage{multicol}
\usepackage{mathtools}
\usepackage{amsmath}
\usepackage[a4paper, total={6in, 8in}]{geometry}
\usepackage{enumerate}
\usepackage{sectsty}
\usepackage[none]{hyphenat}
\usepackage{setspace}
\usepackage{cuted}
\usepackage{nameref}
\usepackage[utf8]{inputenc}

\DeclareMathOperator*{\argmin}{argmin}
\DeclareMathOperator*{\argmax}{argmax}

%--------------------------------------------------------------------------------
% Notations

\usepackage[nopostdot,  style=long3col, nonumberlist, toc]{glossaries}

\newglossary{symbols}{sym}{sbl}{Nomenclature}

\makeglossary


\newglossaryentry{city_source}
{
type=symbols,
name={\ensuremath{i}},
sort={01},
description={Source City}
}

\newglossaryentry{city_destination}
{
type=symbols,
name={\ensuremath{j}},
sort={01},
description={Destination City}
}

\newglossaryentry{city_set}
{
type=symbols,
name={\ensuremath{I}},
sort={02},
description={Set of cities}
}

\newglossaryentry{city_count}
{
type=symbols,
name={\ensuremath{N}},
sort={02},
description={Total number of cities (\gls{city_set})}
}

\newglossaryentry{distance}
{
type=symbols,
name={\ensuremath{D_{ij}}},
sort={02},
description={Distance between \gls{city_source} and \gls{city_destination}}
}

\newglossaryentry{dv_xij}
{
type=symbols,
name={\ensuremath{X_{ij}}},
sort={03},
description={Binary falg, sales man travels from \gls{city_source} and \gls{city_destination}}
}

\newglossaryentry{av_ui}
{
type=symbols,
name={\ensuremath{U_{i}}},
sort={04},
description={Integer, Artificial  variable for \gls{city_source}}
}

\newglossaryentry{av_uj}
{
type=symbols,
name={\ensuremath{U_{j}}},
sort={04},
description={Integer, Artificial  variable for \gls{city_destination}}
}
%--------------------------------------------------------------------------------
%Title and section font sizes

\parttitlefont{\Large}
\sectionfont{\large}

%--------------------------------------------------------------------------------

\renewenvironment{abstract}
 {\par\noindent\textbf{\Large\abstractname}\\ \ignorespaces}
 {\par\medskip}


%--------------------------------------------------------------------------------
% Title

\title{Delivery schedule optimization for Direct Store Delivery categories}
\author{
Diptesh Basak\\
\and
Ishita Roy\\
}
\date{\today}

\begin{document}

\maketitle

%--------------------------------------------------------------------------------

\pagebreak

\begin{abstract}
\hrule
\hfill
\doublespacing
\newline Replenishment at Target in its current state focusses on filling up the sales floor capacity (planogram) and stocking future demand in the backroom. 

\end{abstract}

\pagebreak

\tableofcontents

\listoftables

\pagebreak

\hrule

\begin{multicols}{2}

\section{Travelling Salesman Problem}
\label{section:obj}

The travelling salesman problem (TSP) is about given a list of cities and the distances between each pair of cities, what is the shortest possible route that visits each city exactly once and returns to the origin city

\begin{flushleft}
{
\emph{Data Used:}
\begin{flalign}
\nonumber & \gls{city_source} \in \gls{city_set} & \\
\nonumber & \gls{city_destination} \in \gls{city_set} & \\
\nonumber  & \gls{distance} &
\end{flalign}
}



\emph{Decision variables:}\\

\vspace{0.3cm}
$
 \hspace{0.2cm}
 \gls{dv_xij}=
 \begin{cases}
 1,&\text{if sales man travels from \gls{city_source} to \gls{city_destination} in optimal route} \\
 0, & \text{otherwise}
 \end{cases}
 \vspace{0.2cm}
$

$
 \hspace{0.2cm}
 \vspace{0.2cm}
 \gls{av_ui} \in Integer
$

$
 \hspace{0.2cm}
 \gls{av_uj} \in Integer
 \vspace{0.2cm}
$
\vspace{0.2cm}

\end{flushleft}

\emph{Objective function:}

\begin{equation}
\min \sum_{\gls{city_source} \in \gls{city_set}} \sum_{\gls{city_destination} \in  \gls{city_set}} \gls{dv_xij} \times \gls{distance} 
\end{equation}

\emph{s.t.}
\\
\vspace{2mm}
Each node should be entered and exited exactly once

\begin{flalign}
& \sum_{\gls{city_source}} \gls{dv_xij} = 1 & \forall \gls{city_destination}
\end{flalign}

\begin{flalign}
& \sum_{\gls{city_destination}} \gls{dv_xij} = 1 & \forall \gls{city_source}
\end{flalign}

Eliminate subtours
\begin{flalign}
& \gls{av_ui} -  \gls{av_uj} + \gls{city_count} \times \gls{dv_xij} = \gls{city_count} - 1 & \forall \gls{city_source} \in 1,2..\gls{city_count}-1 \hspace{2mm} \gls{city_destination} \in 2, 3 .. \gls{city_count}
\end{flalign}
\end{multicols}



%--------------------------------------------------------------------------------
% Notations

\pagebreak

\hrule

\printglossaries

%--------------------------------------------------------------------------------

\end{document}

%--------------------------------------------------------------------------------
